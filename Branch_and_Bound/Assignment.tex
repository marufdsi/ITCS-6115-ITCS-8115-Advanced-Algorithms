\documentclass[11pt, conference, onecolumn]{IEEEtran}
\IEEEoverridecommandlockouts
\usepackage{amsmath,amssymb,amsfonts}
\usepackage{graphicx}
\usepackage{textcomp}
\usepackage{xcolor}
\usepackage{algorithm,algpseudocode}
\algrenewcommand\algorithmicindent{0.9em}%
\usepackage{soul}
\usepackage{xspace}
\usepackage{subfigure}
\pagestyle{plain}
\DeclareMathOperator*{\argminA}{arg\,min} % Jan Hlavacek
\DeclareMathOperator*{\argmaxA}{arg\,max} % Jan Hlavacek

\newcommand{\todo}[1]{\color{red}\textbf{\hl{#1}}\color{black}\xspace}
%\newcommand{\todo}[1]{}
\newcommand{\rom}[1]{\expandafter{\romannumeral #1\relax}}

\def\BibTeX{{\rm B\kern-.05em{\sc i\kern-.025em b}\kern-.08em
    T\kern-.1667em\lower.7ex\hbox{E}\kern-.125emX}}
\begin{document}

\title{Branch and Bound presentation }

\maketitle
\section*{Minimize the Maximum Stretch of Job Schedule}
Design strategies to  solve that problem with a branch and bound without relying on ILP formulation. The components are:
\begin{enumerate}

\item what is the problem and why is in important?

\item How are subsets represented and how are partitioning of the solution sapce accomplished?

\item How are the branching decisions made? In which order are the partitions made?

\item How is bounding going to work? Which relaxations can be done? How would the relaxed problems be solved optimally?

\item What order of the traversal seem to make sense for this problem and branching and bounding strategy?

\item Are there symmetries or dominance properties that can be taken advantage of?

\item How would deriving a solution from a particular subset be accomplished?

\end{enumerate}
\end{document}